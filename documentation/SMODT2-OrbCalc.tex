\documentclass[10pt,preprint]{aastex}
%\usepackage{thumbpdf}
\usepackage{fullpage,amssymb,mathtools}
\usepackage{mathrsfs}
\usepackage{color}
\usepackage{natbib}
\bibliographystyle{apj}

\addtolength{\oddsidemargin}{-.5in}
\addtolength{\evensidemargin}{-.5in}
\addtolength{\textwidth}{0.5in}

\title{Models Used in 3D Orbit Simulations}
\author{Kyle Mede}
\date{\today}

\begin{document}

\maketitle

\section{Model}
THe goal is to calculate the values for $x$, $y$, and $v$ at given epoch ($t$).  This will be done in three stages, calculate the anomalies ($M$, $E$ and $\theta$) and then use them to calculate $x$ and $y$, then finally $RV$.


%-----------------------------------------------------------------------------------
\subsection{Calculate the Anomalies}

First we need to calculate the True Anomaly ($\theta$) as it is used by the equations for both the astrometry and radial velocity models.

To do that we will need to calculate the Mean Motion, $n$, time of current epoch($t$), time of last periapsis($T$), and time of transit center relative to last periapsis ($T_c$) to get the Mean Anomaly:
\begin{subequations}\label{eq:n-phase-M}
\begin{align}
\label{eq:n}
n& = \frac{2\pi}{P_{years}} \\
\label{eq:phase}
phase& = \frac{Tc-T}{P_{days}} \\
\label{eq:Ma}
M& = n \bigg( \frac{(t-T)}{DaysPerYear} \bigg)+(phase)2\pi
\end{align}
\end{subequations}
Note: in cases where there is only astrometry data, $T_c==T$.

This resulting equation for $M$ can be simplified.
\begin{subequations}
\begin{align}
\label{eq:Mreduced-a}
M& = \frac{2\pi}{P_{years}} \bigg( \frac{(t-T)}{DaysPerYear}\bigg)+ \bigg(\frac{Tc-T}{P_{days}}\bigg)2\pi\\
\label{eq:Mreduced-b}
M& = 2\pi\bigg[\frac{1}{P_{years}}\bigg( \frac{(t-T)}{DaysPerYear}\bigg)+\bigg(\frac{Tc-T}{P_{years}\times DaysPerYear}\bigg)\\
\label{eq:Mreduced-c}
M& = \frac{2\pi}{P_{years}\times DaysPerYear}[(t-T)+(Tc-T)]\\
\label{eq:Mreduced-d}
M& = \frac{2\pi(t-2T+T_c)}{P_{years}\times DaysPerYear}
\end{align}
\end{subequations}
working out to units of radians for $M$.

The relation between the Eccentric Anomaly, E, and the Mean Anomaly, M, is a transcendental equation and must be solved using numerical methods, shown below.
\begin{equation}\label{eq:4.1.3}
M = E - e\times\sin(E)
\end{equation}
In order to obtain the solution for E the fastest using the Newton's loop and (\ref{eq:4.1.4}), the closest guess of E should be used as the initial value of E$\_$last.
  This also helps to avoid ending up with one of the wrong solutions in the
  cases where there are multiple crossings of the two functions that make up
  equation (\ref{eq:4.1.3}).  A suggested initial guess, that we found to work well, is given by (\ref{eq:4.1.3.5}) as was recommended in \citet{Argyle}.

\begin{equation}\label{eq:4.1.3.5}
E_0 = M+e\times\sin(M) + \frac{e^2}{2M}sin(2M)
\end{equation}

Newton's method to calculate E:
  
\begin{equation}\label{eq:4.1.4}
E = E^{'} - \bigg[\frac{E^{'} - e \times \sin(E^{'}) - M}{1.0 - e \times \cos(E^{'})}\bigg]
\end{equation}
The loop completes when $E$ and $E^{'}$ are the same to 10 decimal places.  It is also checked to ensure it satisfies the original equation (\ref{eq:4.1.3}) with similar precision.  The maximum value of e possible was found to be ~0.98, as precisional rounding issues caused division by zero above this.\\

Use the resultant $E$ to calculate the True Anomaly:
\begin{subequations}
\begin{align}
\label{eq:4.1.5a}
\theta^{'}& = \arccos \bigg( \frac{\cos(E) - e}{1.0 - e \times \cos(E)} \bigg)\\
\label{eq:4.1.5b}
\theta& = \left\{ \begin{array}{l l} \theta^{'}& \quad \text{ if $E\leq 180^{\circ}$}\\ 360^{\circ}  - \theta^{'}& \quad \text{ if $E > 180^{\circ}$} \end{array}\right.
\end{align}
\end{subequations}

Equation (\ref{eq:4.1.5a}) has one unfortunate attribute, as the Eccentric Anomaly grows over $\pi$ (180$^{\circ}$) the resulting value for the True Anomaly goes down, rather than up as should happen.  Thus, to solve this problem the conditional statements of (\ref{eq:4.1.5b}) are applied.\\ 

%-----------------------------------------------------------------------------------
\subsection{calculate $x$ and $y$}
{\color{red} DISCUSS IN  DI FITTING IT IS THE COMBINED ORBIT BEING FIT!!}

First we will use Kepler's third law to calculate the total semi-major axis ($a=a_1+a_2$):
\begin{equation}\label{eq:K3}
a = \bigg[\frac{P^2G(M_1+M_2)}{4\pi^2} \bigg]^{(1/3)}
\end{equation}

The Thiele-Innes method of solving for the orbital elements of binary systems was first found by \citet{Thiele}, and advanced with the inclusion of the Innes constants formulated by \citet{Van}.  This approach has been mainstream ever since, and the equations to find the ephemeris are given below and can be found in \citet{aitken}, \citet{binnendijk} and \citet{heintz}.

\begin{subequations}
\begin{align}\label{eq:24a}
A& = a[cos(\Omega)cos(\omega)-sin(\Omega)sin(\omega)cos(i)]\\
\label{eq:24b}
B& = a[sin(\Omega)cos(\omega)+cos(\Omega)sin(\omega)cos(i)]\\
\label{eq:24c}
F& = a[-cos(\Omega)sin(\omega)-sin(\Omega)cos(\omega)cos(i)]\\
\label{eq:24d}
G& = a[-sin(\Omega)sin(\omega)+cos(\Omega)cos(\omega)cos(i)]
\end{align}
\end{subequations}
with $a$ being in units of [$\arcsec$], and the angles ($\Omega$, $\omega$ and $i$) in units of radians.

The $x$ and $y$ components of the location on the apparent ellipse are found using:
\begin{subequations}
\begin{align}\label{eq:28-1a}
x& = AX+FY\\
\label{eq:28-1b}
y& = BX + GY
\end{align}
\end{subequations}


%-----------------------------------------------------------------------------------
\subsection{calculate $v$}

The mass ratio can be calculated and used to determine the individual semi-major axis values for each object's orbit as follows:
\begin{subequations}
\begin{align}
\label{eq:massRatio}
x& = \frac{Mass_2}{Mass_1}\\
\label{eq:massRatio-a1}
a_1& = \frac{a}{1+x}\\
\label{eq:massRatio-a2}
a_2& = \frac{a1}{x}
\end{align}
\end{subequations}

The measured velocity of the primary star due to the motion of the companion can be given by:
\begin{subequations}
\begin{align}
\label{eq:v-a}
v =& \frac{2\pi a_1sin(i)}{P\sqrt{1-e^2}}[cos(\theta+\omega)+e cos(\omega)]\\
\label{eq:v-b}
v =& \frac{2\pi \frac{a}{1+x}sin(i)}{P\sqrt{1-e^2}}[cos(\theta+\omega)+e cos(\omega)]\\
\label{eq:v-c}
v =& K[cos(\theta+\omega)+e\times cos(\omega)]
\end{align}
\end{subequations}

{\color{red} DISCUSS ISSUES OF WHAT OBJECT IS BEING MEASURED AND WHAT OBJECT'S ORBIT IS BEING FIT!!}

%--------------------------------------------------------------------------------------------------------------------------------------
\pagebreak
\bibliography{SMODT-citations}
\clearpage

\end{document}